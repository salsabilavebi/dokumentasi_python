
\documentclass[12pt]{article} % use larger type; default would be 10pt

\usepackage[utf8]{inputenc}
\usepackage{geometry} % to change the page dimensions
\geometry{a4paper}
\usepackage{graphicx} 
%%% PACKAGES
\usepackage{booktabs} % for much better looking tables
\usepackage{array} % for better arrays (eg matrices) in maths
\usepackage{paralist} % very flexible & customisable lists (eg. enumerate/itemize, etc.)
\usepackage{verbatim} % adds environment for commenting out blocks of text & for better verbatim
\usepackage{subfig} % make it possible to inc
\usepackage{fancyhdr} % This should be set AFTER setting up the page geometry
\usepackage{epstopdf}
\usepackage{listings}
\usepackage{authblk}
\usepackage{amssymb}
\usepackage{amsmath}

\pagestyle{fancy} % options: empty , plain , fancy
\renewcommand{\headrulewidth}{0pt} % customise the layout...
\lhead{}\chead{}\rhead{}
\lfoot{}\cfoot{\thepage}\rfoot{}

%%% SECTION TITLE APPEARANCE
\usepackage{sectsty}
\allsectionsfont{\sffamily\mdseries\upshape} 

\lstset{
	language=Python,
	showstringspaces=false,
	columns=flexible,
	breaklines=true,
	tabsize=3
}

 \sectionfont{\fontsize{12}{12}\selectfont}

\begin{document}

Nama	: Salsabia Vebi Natasya \\
NPM		: 1194066 \\
Kelas	: D4TI-3B \\
	
\section{Cara menampilkan output}
\begin{lstlisting}
	nama = "Salsabila Vebi"
	print "Hello",nama
\end{lstlisting}
	
\section {Cara mendapatkan hostname system}
\begin{lstlisting}
	import socket
	print(socket.gethostname())
\end{lstlisting}
	
\section{ Menulis data dictionary ke CSV}
\begin{lstlisting}
	import csv
		
	with open('contacts.csv', mode='a') as csv_file:
	# menentukan label
	fieldnames = ['NO', 'NAMA', 'TELEPON']
		
	# membuat objek writer
	writer = csv.DictWriter(csv_file, fieldnames=fieldnames)
		
	# menulis baris ke file CSV
	writer.writeheader()
	writer.writerow({'NO': '10', 'NAMA': 'Salsa', 'TELEPON': '0821194512'})
	writer.writerow({'NO': '11', 'NAMA': 'Vebi', 'TELEPON': '08213218211'})
		
	print("Writing Done!")
\end{lstlisting}
	
	
\section{cara mendapatkan key dan value dan menggabungkan dalam daftar}
\begin{lstlisting}
	data = [{"name":"Anne", "followers":["Brian"]}, {"name":"Cindy", "followers":["Brian","G osh","Anne"]},{"name":"Dave", "followers":[]}]
		
	output : [{"name": ["Brian"] , "follows":["Anne","Cindy"]},...] etc...
		
	my code from now :
		
	from operator import itemgetter
		
	data = data = [{"name":"Anne", "followers":["Brian"]}, {"name":"Cindy", "followers":["Brian","Gosh","Anne"]},{"name":"Dave", "followers":[]}]
		
	x = list(map(itemgetter('followers'), data))
	y = list(map(itemgetter('name') ,data))
	print("name : " + str(x), " follow : " + str(y))
\end{lstlisting}
	
\section{Cara menemukan semua kombinasi dari 3 kerangka data dan mengembalikannya sebagai daftar}
\begin{lstlisting}
	>>> from datar.all import f, tibble, bind_cols, expand, nesting
	>>> 
	>>> df1 = tibble(
	...     name=["John", "Nick", "Eric"], job=["engineer", "architect", "deisgner"]
	... )
	>>> df2 = tibble(
	...     city=["London", "Montresor", "Esslingen"],
	...     bigness=["captical", "villege", "town"],
	... )
	>>> df3 = tibble(
	...     street=["street1", "street2", "street3"],
	...     population=["high", "low", "average"],
	... )
	>>> 
	>>> df = bind_cols(df1, df2, df3)
	>>> df >> expand(
	...     nesting(f.name, f.job),
	...     nesting(f.city, f.bigness),
	...     nesting(f.street, f.population),
	... )
\end{lstlisting}
	
\section {Cara membuat list}
\begin {lstlisting}
	list1 = ['kimia', 'fisika', 1993, 2017]
	list2 = [1, 2, 3, 4, 5 ]
	list3 = ["a", "b", "c", "d"]
\end {lstlisting}
	
\section {Menggabungkan dua buah list}
\begin{lstlisting}
	listone = [1, 2, 3]
	listtwo = [4, 5, 6]
		
	joinedlist = listone + listtwo
\end{lstlisting}
	
\section {Cara mengakses nilai dalam list}
\begin{lstlisting}
	list1 = ['fisika', 'kimia', 1993, 2017]
	list2 = [1, 2, 3, 4, 5, 6, 7 ]
		
	print ("list1[0]: ", list1[0])
	print ("list2[1:5]: ", list2[1:5])
\end{lstlisting}
	
\section{Cara mendapatkan waktu Saat ini}
\begin {lstlisting}
	import time;
	
	localtime = time.localtime(time.time())
	print "Waktu lokal saat ini :", localtime
\end{lstlisting}

\section{Cara memnggunakan Waktu selektif pada database python}
\begin{lstlisting}
	df.sort_values('Submit Date').drop_duplicates(subset=['customer'], keep='last')
\end{lstlisting}

\section {Convert date ke datetime pada python}
\begin{lstlisting}
	from datetime import date
	from datetime import datetime
		
	dt = datetime.combine(date.today(), datetime.min.time())
\end{lstlisting}

\section{Cara menggunakan waktu Tick}
\begin {lstlisting}
	import time; 
	
	ticks = time.time()
	print "Berjalan sejak 12:00am, January 1, 1970:", ticks
\end{lstlisting}
	
\section{Menggunakan fungsi create}
\begin{lstlisting}
	def create_contact():
	clear_screen()
	with open(csv_filename, mode='a') as csv_file:
	fieldnames = ['NO', 'NAMA', 'TELEPON']
	writer = csv.DictWriter(csv_file, fieldnames=fieldnames)
	
	no = input("No urut: ")
	nama = input("Nama lengkap: ")
	telepon = input("No. Telepon: ")
	
	writer.writerow({'NO': no, 'NAMA': nama, 'TELEPON': telepon})    
	print("Berhasil disimpan!")
	
	back_to_menu()
\end{lstlisting}

\section{Membuat fungsi search}
\begin{lstlisting}
	def search_contact():
	clear_screen()
	contacts = []
	
	with open(csv_filename, mode="r") as csv_file:
	csv_reader = csv.DictReader(csv_file)
	for row in csv_reader:
	contacts.append(row)
	
	no = input("Cari berdasrakan nomer urut> ")
	
	data_found = []
	
	# mencari contact
	indeks = 0
	for data in contacts:
	if (data['NO'] == no):
	data_found = contacts[indeks]
	
	indeks = indeks + 1
	
	if len(data_found) > 0:
	print("DATA DITEMUKAN: ")
	print(f"Nama: {data_found['NAMA']}")
	print(f"Telepon: {data_found['TELEPON']}")
	else:
	print("Tidak ada data ditemukan")
	back_to_menu()
	\end{lstlisting}
	
	\section{Membuat fungsi edit}
	\begin{lstlisting}
	def edit_contact():
	clear_screen()
	contacts = []
	
	with open(csv_filename, mode="r") as csv_file:
	csv_reader = csv.DictReader(csv_file)
	for row in csv_reader:
	contacts.append(row)
	
	print("NO \t NAMA \t\t TELEPON")
	print("-" * 32)
	
	for data in contacts:
	print(f"{data['NO']} \t {data['NAMA']} \t {data['TELEPON']}")
	
	print("-----------------------")
	no = input("Pilih nomer kontak> ")
	nama = input("nama baru: ")
	telepon = input("nomer telepon baru: ")
	
	# mencari contact dan mengubah datanya
	# dengan data yang baru
	indeks = 0
	for data in contacts:
	if (data['NO'] == no):
	contacts[indeks]['NAMA'] = nama
	contacts[indeks]['TELEPON'] = telepon
	indeks = indeks + 1
	
	# Menulis data baru ke file CSV (tulis ulang)
	with open(csv_filename, mode="w") as csv_file:
	fieldnames = ['NO', 'NAMA', 'TELEPON']
	writer = csv.DictWriter(csv_file, fieldnames=fieldnames)
	writer.writeheader()
	for new_data in contacts:
	writer.writerow({'NO': new_data['NO'], 'NAMA': new_data['NAMA'], 'TELEPON': new_data['TELEPON']}) 
	
	back_to_menu()
\end{lstlisting}

\section{Menggunakan fungsi delete}
\begin{lstlisting}
	def delete_contact():
	clear_screen()
	contacts = []
	
	with open(csv_filename, mode="r") as csv_file:
	csv_reader = csv.DictReader(csv_file)
	for row in csv_reader:
	contacts.append(row)
	
	print("NO \t NAMA \t\t TELEPON")
	print("-" * 32)
	
	for data in contacts:
	print(f"{data['NO']} \t {data['NAMA']} \t {data['TELEPON']}")
	
	print("-----------------------")
	no = input("Hapus nomer> ")
	
	# mencari contact dan mengubah datanya
	# dengan data yang baru
	indeks = 0
	for data in contacts:
	if (data['NO'] == no):
	contacts.remove(contacts[indeks])
	indeks = indeks + 1
	
	# Menulis data baru ke file CSV (tulis ulang)
	with open(csv_filename, mode="w") as csv_file:
	fieldnames = ['NO', 'NAMA', 'TELEPON']
	writer = csv.DictWriter(csv_file, fieldnames=fieldnames)
	writer.writeheader()
	for new_data in contacts:
	writer.writerow({'NO': new_data['NO'], 'NAMA': new_data['NAMA'], 'TELEPON': new_data['TELEPON']}) 
	
	print("Data sudah terhapus")
	back_to_menu()
\end{lstlisting}

\section{ Menggunakan Daftar Lists sebagai Tumpukan Stacks}
\begin{lstlisting}
	>>> stack = [3, 4, 5]
	>>> stack.append(6)
	>>> stack.append(7)
	>>> stack
	[3, 4, 5, 6, 7]
	>>> stack.pop()
	7
	>>> stack
	[3, 4, 5, 6]
	>>> stack.pop()
	6
	>>> stack.pop()
	5
	>>> stack
	[3, 4]
\end{lstlisting}

\section{membuat menu dari input pengguna}
\begin{lstlisting}
	def functions():
	max_length = int(input("how many products in your card ? : "))
	select_function = input("press 1 to add product names to the menu or 2 to assign prices : ")
	select_function = int(select_function)
	products = []
	while select_function == 1 and len(products) != max_length :
	items = input("enter product name : ")
	items = items.split()
	products.append(items)
	if len(products) == max_length :
	select_function == 2
	price = []
	while select_function ==2 and len(price) != max_length :
	items = input("enter product price : ")
	items = items.split()
	price.append(items)
	menu = dict(zip(products,price))
	print(menu)
\end{lstlisting}

\section{Cara menggunakan operator relasi sama dengan}
\begin{lstlisting}
		lulus = raw_input("Apakah kamu lulus? [ya/tidak]: ")
		
		if lulus == "tidak":
		print("Kamu harus mengulang ujian")
\end{lstlisting}

\section{Penggunaan If/Else}
\begin{lstlisting}
	umur = input("Berapa umur kamu: ")
	
	if umur >= 17:
	print("Kamu boleh membuat KTP")
	else:
	print("Kamu belum boleh membuat KTP")
\end{lstlisting}

\section{Parsing XML di Python}
\begin{lstlisting}
	import xml.dom.minidom as minidom
	
	def main():
	# gunakan fungsi parse() untuk me-load xml ke memori 
	# dan melakukan parsing
	doc = minidom.parse("mahasiswa.xml")
	
	# Cetak isi doc dan tag pertamanya
	print doc.nodeName
	print doc.firstChild.tagName
	
	
	if __name__ == "__main__":
	main()

\end{lstlisting}

\section{Mengakses nilai dalam dict python}
\begin{lstlisting}
	dict = {'Name': 'Salsa', 'Age': 20, 'Class': 'First'}
	print ("dict['Name']: ", dict['Name'])
	print ("dict['Age']: ", dict['Age'])
\end{lstlisting}

\section{Cara menggunakan Lambda Expression}
\begin{lstlisting}
	greeting = lambda name: print(f"Hello, {name}")
	sapa = greeting
	greeting("Andi")
	sapa("Salsa")
\end{lstlisting}

\section{Cara membuat dictionary}
\begin{lstlisting}
Test = {
	"nama": "Salsabila Vebi Natasya",
	"umur": 20,
	"hobi": ["coding", "membaca", "tidur"],
	"menikah": False,
	"sosmed": {
		"facebook": "salsa",
		"twitter": "@salsa"
	} 
}
\end{lstlisting}

\section{Mengubah nilai item Dictionary}
\begin{lstlisting}
	skill = {
		"utama": "Tidur",
		"lainnya": ["PHP","Java", "HTML"]
	}
	
	# Mencetak isi skill utama
	print(skill["utama"])
	
	# mengubah isi skill utama
	skill["utama"] = "Mencetak gol"
	
	# Mencetak isi skill utama
	print(skill["utama"])
\end{lstlisting}

\section{Mengambil panjang atau length Dictionary}
\begin{lstlisting}
	books = {
		"python": "Menguasai Python dalam 2028 jam",
		"java": "Tutorial Belajar untuk Pemula",
		"php": "Membuat aplikasi web dengan PHP"
	}
	
	# mencetak jumlah data yang ada di dalam dictionary
	print("total buku: %d" % len(books))
\end{lstlisting}

\section{Penggunaan Range pada perulangan for}
\begin{lstlisting}
	for nomer in range(10):
	print "mahasiswa-" + str(nomer)
\end{lstlisting}


\section{mendapatkan 2 nilai per item }
\begin{lstlisting}
	n = 100000000
	l = int(n/6)
	f1 = lambda x: (6*x)-1
	f3 = lambda x: (6*x)+1
	primeCandidate = [f(i) for i in range(1,l+1) for f in (f1,f3)]
\end{lstlisting}

\section{Cara menggunakan generator expressions}
\begin{lstlisting}
	def squares(length):
	for n in range(length):
	yield n ** 2
\end{lstlisting}

\section{Mendapatkan semua kemungkinan urutan boolean untuk panjang daftar tertentu}
\begin{lstlisting}
	from itertools import product
	
	[seq for seq in product((True, False), repeat=3)][1:-1]
\end{lstlisting}

\section{Cara mendapatkan nama kolom kerangka data dari nilai dalam array numpy}
\begin{lstlisting}
	threshold = .5
	
	for j in range(loads.shape[1]):
	print(df.columms[loads[:,j]>threshold])
\end{lstlisting}

\section{Menghapus duplikat dari korelasi matrix}
\begin{lstlisting}
	def view_corr(df):
	df = df.unstack()
	corr_f = df.sort_values(kind="quicksort", ascending=False)
	corr_f = corr_f.dropna().drop_duplicates() # <<<---- here
	corr_f = corr_f[corr_f<1]
	print(corr_f[corr_f>0.10])
\end{lstlisting}

\section{Pengambilan panjang Tuple}
\begin{lstlisting}
	# Membuat Tuple
	hari = ('Senin', 'Selasa', 'Rabu', 'Kamis', 'Jum\'at', 'Sabtu', 'Minggu')
	
	# Mengambil panjang tuple hari
	print("Jumlah hari: %d" % len(hari))
\end{lstlisting}

\section{Perulangan Nested Loop}
\begin{lstlisting}
	i = 2
	while(i < 100):
	j = 2
	while(j <= (i/j)):
	if not(i%j): break
	j = j + 1
	if (j > i/j) : print(i, " is prime")
	i = i + 1
	
	print("Good bye!")
\end{lstlisting}

\section {Membuat list dengan besaran yang ditentukan}
\begin{lstlisting}
	>>> Z = [Yes] * 3
	>>> Z
	[Yes, Yes, Yes]
\end{lstlisting}

\section {Melakukan pengecekan list kosong}
\begin{lstlisting}
	if not a:
	print("List is empty")
\end{lstlisting}

\section {Menulis data JSON ke file}
\begin{lstlisting}
	import json
	with open('data.json', 'w') as f:
	json.dump(data, f)
\end{lstlisting}

\section{Menangani Exception}
\begin{lstlisting}
	while True:
	try:
	x = int(input("Please enter a number: "))
	break
	except ValueError:
	print("Oops!  That was no valid number.  Try again...")

\end{lstlisting}

\section{Mengimport semua submodul}
\begin{lstlisting}
	import * from sound.effect
\end{lstlisting}

\section{menghapus item dari list}
\begin{lstlisting}
	>>> a = [-1, 1, 66.25, 333, 333, 1234.5]
	>>> del a[0]
	>>> a
	[1, 66.25, 333, 333, 1234.5]
	>>> del a[2:4]
	>>> a
	[1, 66.25, 1234.5]
	>>> del a[:]
	>>> a
	[]
\end{lstlisting}

\section{menghapus seluruh variabel}
\begin{lstlisting}
	>>> del a
\end{lstlisting}

\section{Mengulang urutan secara terbalik}
\begin{lstlisting}
	>>> for i in reversed(range(1, 10, 2)):
	print(i)
	
	9
	7
	5
	3
	1
\end{lstlisting}

\section{Perulangan if}
\begin{lstlisting}
	a = 8
	b = 10
	if b > a:
	print("b lebih besar dari a")
\end{lstlisting}

\section{Perulangan While}
\begin{lstlisting}
	count = 0
	while (count < 9):
	print ("The count is: ", count)
	count = count + 1
	
	print ("Good bye!")
\end{lstlisting}

\section{Perulangan For}
\begin{lstlisting}
	angka = [1,2,3,4,5]
	for x in angka:
	print(x)
	
	buah = ["nanas", "apel", "jeruk"]
	for makanan in buah:
	print ("Saya suka makan", makanan)
\end{lstlisting}

\section{Penggunaan Variabel}
\begin{lstlisting}
	nama = "John Doe" l
	print(nama)
	
	#nilai dan tipe data dalam variabel  dapat diubah
	umur = 20               #nilai awal
	print(umur)             #mencetak nilai umur
	type(umur)              #mengecek tipe data umur
	umur = "dua puluh satu" #nilai setelah diubah
	print(umur)             #mencetak nilai umur
	type(umur)              #mengecek tipe data umur
	
	namaDepan = "Budi"
	namaBelakang = "Susanto"
	nama = namaDepan + " " + namaBelakang
	umur = 22
	hobi = "Berenang"
	print("Biodata\n", nama, "\n", umur, "\n", hobi)
	
	#contoh variabel lainya
	inivariabel = "Halo"
	ini_juga_variabel = "Hai"
	_inivariabeljuga = "Hi"
	inivariabel222 = "Bye" 
	
	panjang = 10
	lebar = 5
	luas = panjang * lebar
	print(luas)
\end{lstlisting}

\section{Membuat Instance Object}
\begin{lstlisting}
	emp1 = Employee("Zara", 2000)
	emp2 = Employee("Manni", 5000)
\end{lstlisting}

\section {Menginstall package menggunakan pip berdasarkan requirements.txt}
\begin{lstlisting}
	pip install -r /path/to/requirements.txt
\end{lstlisting}

\section{Penggunaan dari *args dan **kwargs}
\begin{lstlisting}
	# membuat fungsi dengan parameter *args
	def kirim_sms(*nomer):
	print nomer
	
	# membuat fungsi dengan parameter **kwargs
	def tulis_sms(**isi):
	print isi
	
	# Pemanggilan fungsi *args
	kirim_sms(123, 888, 4444)
	
	# pemanggilan fungsi **kwargs
	tulis_sms(tujuan=123, pesan="apa kabar")
\end{lstlisting}

\section{Cara memilihi item secara acak dari daftar}
\begin{lstlisting}
	import random
	
	foo = ['battery', 'correct', 'horse', 'staple']
	secure_random = random.SystemRandom()
	print(secure_random.choice(foo))
\end{lstlisting}

\section {Import files dari folder berbeda}
\begin{lstlisting}
	# some_file.py
	import sys
	# insert at 1, 0 is the script path (or '' in REPL)
	sys.path.insert(1, '/path/to/application/app/folder')
	
	import file
\end{lstlisting}

\section {operasi pada array menggunakan NumPy}
\begin{lstlisting}
	import numpy as np
	a = np.array([1, 2, 3])
	f = np.array([1.1, 2.2, 3.3])
	a*f
\end{lstlisting}

\section {Membuat DataFrame Menggunakan Pandas}
\begin{lstlisting}
	import pandas as pd
	data = {'kota' : ['semarang', 'semarang', 'semarang', 'bandung', 'bandung', bandung],
		'tahun' : ['2016', '2017', '2018', '2016', '2017','2018'],
		'populasi': [1.5, 2.1, 3.2, 2.3, 3.2, 4.5]}
	frame = pd.DataFrame(data)
	frame
\end{lstlisting}

\section {Plotting dasar menggunakan Matplotlib}
\begin{lstlisting}

	import numpy as np
	import matplotlib.pylab as pl
	x = np.array([1,2,3,4,5], float) # membuat nilai array sumbu x
	y = np.array([1,4,9,16,25], float) # membuat nilai array sumbu y
	pl.plot(x,y) # menggunakan pylab untuk memplot x dan y
	[<matplotlib.lines.Line2D object at 0x0000000006064CC0>]
	pl. show() # menampilkan hasil plot pada layar
\end{lstlisting}

\section {Plotting Histogram menggunakan Matplotlib}
\begin{lstlisting}

	import matplotlib.pylab as pl
	pl.show()
	data = np.random.normal(5., 3., 1000)
	# membuat histogram dari data array
	
	pl.hist(data)
	(array([ 1., 13., 43., 146., 244., 251., 185., 91., 20., 6.]),
	array([ -6.05563989, -3.93179046, -1.80794103, 0.3159084 ,
	2.43975783, 4.56360726, 6.68745669, 8.81130612,
	10.93515555, 13.05900498, 15.18285441]), <a list of 10 Patch objects>)
	pl.xlabel("data")
	<matplotlib.text.Text object at 0x000000000B6969B0>
	pl.show()
\end{lstlisting}

\section {Multiple Plotting Dalam Sebuah Kanvas}
\begin{lstlisting}
	
	import matplotlib.pylab as pl
	fig1 = pl.figure(1)
	pl.subplot(211)
	<matplotlib.axes._subplots.AxesSubplot object at 0x0000000005229320>
\end{lstlisting}

\section {Plotting Data Dalam Suatu File}
\begin{lstlisting}

	import matplotlib.pylab as pl
	data = np.loadtxt('databohongan.txt')
	pl.plot(data[:,0], data[:,1], 'bo')
	[<matplotlib.lines.Line2D object at 0x0000000004A78E10>]
	pl.xlabel('sumbu x')
	<matplotlib.text.Text object at 0x0000000004913278>
	pl.ylabel('sumbu y')
	<matplotlib.text.Text object at 0x00000000049D7780>
	pl.title('Plotting Data ASCII')
	<matplotlib.text.Text object at 0x0000000004A087B8>
	pl.xlim(0., 10.)
	(0.0, 10.0)
	pl.show()
\end{lstlisting}

\section {Mencari rata-rata pada sebuah list}
\begin{lstlisting}
	l = [15, 18, 2, 36, 12, 78, 5, 6, 9]
	
	import statistics
	statistics.mean(l)
\end{lstlisting}

\section {Logging}
\begin{lstlisting}
	import logging
	
	logger = logging.getLogger()
	
	def f():
	
	try:
	
	flaky_func()
	
	except Exception:
	
	logger.exception()
	
	raise
\end{lstlisting}

\section {perulangan while dengan inputan}
\begin{lstlisting}
	a = int(input('Masukkan bilangan ganjil lebih dari 50: '))
	
	while a % 2 != 1 or a <= 50:
	a = int(input('Salah, masukkan lagi: '))
	
	print('Benar')
\end{lstlisting}

\section {Menuliskan File}
\begin{lstlisting}
	f = file("baru.txt", "w")
	f.write("Baris pertama")
	f.write("masih di baris pertama")
	f.write("\n masuk ke baris kedua")
	f.close()
\end{lstlisting}

\section {Unzipping files}
\begin{lstlisting}
	import zipfile
	with zipfile.ZipFile(path_to_zip_file, 'r') as zip_ref:
	zip_ref.extractall(directory_to_extract_to)
\end{lstlisting}

\section {Iterasi Pada Array}
\begin{lstlisting}
	a = np.array([1,2,8], int)
	for x in a:
	print x
\end{lstlisting}

\section {Penugasan Berganda pada progres Iterasi Pada Array}
\begin{lstlisting}
	a = np.array([[1,2], [3,4], [5,6]], float)
	for (x, y) in a:
	print x * y
\end{lstlisting}

\section {Menggunakan Fungsi Put pada Array}
\begin{lstlisting}
	a = np.array([0,1,2,3,4,5], float
	b = np.array([6,7,8], float)
	a.put([0,3], b)
	a
\end{lstlisting}

\section {perulangan while untuk list}
\begin{lstlisting}
	listKota = ['Jakarta', 'Surabaya', 'Depok', 'Bekasi', 'Solo', 'Jogjakarta', 'Semarang', 'Makassar']
	
	# bermain index
	i = 0
	while i < len(listKota):
	print(listKota[i])
	i += 1
\end{lstlisting}

\section {Bilangan Acak}
\begin{lstlisting}

	np.random.rand(15)

\end{lstlisting}

\section {Date string ke date object}
\begin{lstlisting}
	import datetime
	datetime.datetime.strptime('24052010', "%d%m%Y").date()
	datetime.date(2010, 5, 24)
\end{lstlisting}

\section {Mencari dan mereplace elemen pada list}
\begin{lstlisting}
	a=[1,2,3,1,3,2,1,1]
	[4 if x==1 else x for x in a]
	[4, 2, 3, 4, 3, 2, 4, 4]
\end{lstlisting}

\section {Perbandingan string dengan case-sensitive}
\begin{lstlisting}
	string1 = 'Hello'
	string2 = 'hello'
	
	if string1.casefold() == string2.casefold():
	print("The strings are the same (case insensitive)")
	else:
	print("The strings are NOT the same (case insensitive)")
\end{lstlisting}

\section {Membuat list kosong dengan besaran yang ditentukan}
\begin{lstlisting}
	l = [None] * 10
	l
	[None, None, None, None, None, None, None, None, None, None]
\end{lstlisting}

\section {Menentukan Beberapa Nilai Sekaligus}
\begin{lstlisting}
	v = ('a', 2, True)
	(x, y, z) = v 
	>>> x
	'a'
	>>> y
	2
	>>> z
	True
\end{lstlisting}

\section {Melakukan trigonometri dasar}
\begin{lstlisting}
	>>> import math
	>>> math.pi 
	3.1415926535897931
	>>> math.sin(math.pi / 2) 
	1.0
	>>> math.tan(math.pi / 4) 
	0.99999999999999989
\end{lstlisting}

\section {filter membuat daftar elemen yang mengembalikan fungsi
benar}
\begin{lstlisting}

	number_list = range(-5, 5)
	less_than_zero = list(filter(lambda x: x < 0, number_list))
	print(less_than_zero)
	
	\end{lstlisting}
	
	\section {Penggunaan Reduce }
	\begin{lstlisting}
	
	product = 1
	list = [1, 2, 3, 4]
	for num in list:
	product = product * num
\end{lstlisting}

\section {Penggunaan Set }
\begin{lstlisting}
	some_list = ['a', 'b', 'c', 'b', 'd', 'm', 'n', 'n']
	duplicates = []
	for value in some_list:
	if some_list.count(value) > 1:
	if value not in duplicates:
	duplicates.append(value)
	print(duplicates)
\end{lstlisting}

\section {Ternary Operators }
\begin{lstlisting}
	nice = True
	personality = ("mean", "nice")[nice]
	print("The cat is ", personality)
	\end{lstlisting}
	
	\section {Penggunaan Map}
	\begin{lstlisting}
	def multiply(x):
	return (x*x)
	def add(x):
	return (x+x)
	funcs = [multiply, add]
	for i in range(5):
	value = list(map(lambda x: x(i), funcs))
	print(value)
	\end{lstlisting}
	
	\section{Menggunakan main loop}
	\begin{lstlisting}
	if __name__ == "__main__":
	while True:
	show_menu()
\end{lstlisting}

\section {perulangan while dengan break}
\begin{lstlisting}
	listKota = [
	'Jakarta', 'Surabaya', 'Depok', 'Bekasi', 'Solo',
	'Jogjakarta', 'Semarang', 'Makassar'
	]
	
	kotaYangDicari = input('Masukkan nama kota yang dicari: ')
	
	i = 0
	while i < len(listKota):
	if listKota[i].lower() == kotaYangDicari.lower():
	print('Ketemu di index', i)
	break
	
	print('Bukan', listKota[i])
	i += 1

\end{lstlisting}

\section {Mencetak exception dengan Python}
\begin{lstlisting}
	except Exception as e: print(e)
\end{lstlisting}

\section {Menggunakan Fungsi Fill}
\begin{lstlisting}
	a = np.array([1,2,8], float)
	>>> a
	array([ 1., 2., 8.])
	a.fill(6)
	>>> a
	array([ 6., 6., 6.])
\end{lstlisting}

\section {Perintah In}
\begin{lstlisting}
	>>> a = np.array([[1,2,3], [4,5,6],[1,2,4]],float)
	>>> 2 in a
	True
	>>> 0 in a
	False
\end{lstlisting}

\section {Fungsi Zeros dan ones}
\begin{lstlisting}
	>>> np.ones((2,3), dtype = float)
	array([[ 1., 1., 1.],
	[ 1., 1., 1.]])
	>>> np.zeros(7, dtype = int)
	array([0, 0, 0, 0, 0, 0, 0])
\end{lstlisting}

\section {mengetahui nilai terendah dan tertinggi dari elemen – elemen dalam
suatu array:}
\begin{lstlisting}
	>>> a = np.array([1,2,8], float)
	>>> a.min()
	1.0
	>>> a.max()
	8.0
\end{lstlisting}


\section {mengurutkan elemen – elemen dalam array}
\begin{lstlisting}
	>>> a = np.array([5,1,4,-2,0], float)
	>>> sorted(a)
	[-2.0, 0.0, 1.0, 4.0, 5.0]
	>>> a.sort()
	>>> a
	array([-2., 0., 1., 4., 5.])
\end{lstlisting}

\section {Memotong array sesuai dengan rentang nilai tertentu}
\begin{lstlisting}
	a = np.array([6,2,5,-1,0], float)
	a.clip(0,4)
	array([ 4., 2., 4., 0., 0.])
\end{lstlisting}

\section {Mengekstraksi elemen – elemen unik dalam array}
\begin{lstlisting}
	a = np.array([1,1,1,2,2,3,4,4,4,4,5,5,5,5,5], float)
	np.unique(a)
	array([ 1., 2., 3., 4., 5.])
\end{lstlisting}

\section {Mengekstraksi elemen – elemen diagonal dalam suatu array}
\begin{lstlisting}
	>>> a = np.array([[1,2,3], [4,5,6], [7,8,9]], float)
	>>> a.diagonal()
	array([ 1., 5., 9.])
\end{lstlisting}

\section {Membandingkan suatu array dengan nilai tunggal}
\begin{lstlisting}
	>>> a = np.array([1,2,8], float)
	>>> a > 2
	array([False, False, True], dtype=bool)
\end{lstlisting}

\section { Menerapkan broadcast dalam fungsi where}
\begin{lstlisting}
	>>> np.where(a > 0, 3, 2)
	array([3, 3, 2])
\end{lstlisting}

\section { Penggunaan Fungsi Nonzero}
\begin{lstlisting}
	>>> a = np.array([[0,1], [1,0]], float)
	>>> a.nonzero()
	(array([0, 1], dtype=int64), array([1, 0], dtype=int64))
\end{lstlisting}

\section { Memeriksa keberadaan nilai NaN (not a number) dan bilangan hingga (finite) dalam suatu array}
\begin{lstlisting}
	>>> a = np.array([1, np.NaN, np.Inf], float)
	>>> a
	array([ 1., nan, inf])
	>>> np.isnan(a)
	array([False, True, False], dtype=bool)
	>>> np.isfinite(a)
	array([ True, False, False], dtype=bool)
\end{lstlisting}

\section { Melakukan Perkalian Titik}
\begin{lstlisting}
	>>> a = np.array([1,2,0], float)
	>>> b = np.array([1,2,8], float)
	>>> np.dot(a,b)
	5.0
\end{lstlisting}

\section { Menghitung Determinan dari suatu matriks}
\begin{lstlisting}
	>>> a = np.array([[1,2,8], [1,2,0], [2,2,0]], float)
	>>> a
	array([[ 1., 2., 8.],
	[ 1., 2., 0.],
	[ 2., 2., 0.]])
	>>> np.linalg.det(a)
	-15.999999999999998
\end{lstlisting}

\section {Mengetahui polinom mana yang menghasilkan akar – akar}
\begin{lstlisting}
	>>> np.poly([-1, 1, 1, 10])
	array([ 1., -11., 9., 11., -10.])
\end{lstlisting}

\section {Menghapus duplikasi pada list}
\begin{lstlisting}
	>>> t = [1, 2, 3, 1, 2, 5, 6, 7, 8]
	>>> t
	[1, 2, 3, 1, 2, 5, 6, 7, 8]
	>>> list(set(t))
	[1, 2, 3, 5, 6, 7, 8]
	>>> s = [1, 2, 3]
	>>> list(set(t) - set(s))
	[8, 5, 6, 7]
\end{lstlisting}

\section {Menentukan Bilangan Ganjil Genap}
\begin{lstlisting}
	number = int(input("masukkan bilangan:"))
	
	if number % 2 == 0:
	print("%i adalah bilangan genap" % number)
	else
	print("%I adalah bilangan ganjil" % number)

\end{lstlisting}

\section {Mengganti Nama File}
\begin{lstlisting}
	import os
	try:
	os.rename('absen.txt', 'daftar-hadir.txt')
	print "Nama file sudah diubah.."
	except (IOError, OSError), e:
	print "proses error karena : ", e

\end{lstlisting}

\section {Penggunaan “elif” pada “if”}
\begin{lstlisting}
	print "Masukkan dua buah angka.."
	print "Dan Anda akan check hubungan kedua angka tersebut"
	angka1 = raw_input("Masukkan angka pertama : ")
	angka1 = int(angka1)
	angka2 = raw_input("Masukkan angka kedua : ")
	angka2 = int(angka2)
	if angka1 == angka2 :
	print "%d sama dengan %d" % (angka1, angka2)
	elif angka1 != angka2 :
	print "%d tidak sama dengan %d" % (angka1, angka2)
	elif angka1 < angka2 :
	print "%d kurang dari %d" % (angka1, angka2)
	elif angka1 > angka2 :
	print "%d lebih dari %d" % (angka1, angka2)
	elif angka1 <= angka2 :
	print "%d kurang dari sama dengan %d" % (angka1, angka2)
	elif angka1 >= angka2 :
	print "%d lebih dari sama dengan %d" % (angka1, angka2)

\end{lstlisting}

\section {Membuat teks yang rata kiri dan/atau rata kanan dalam string}
\begin{lstlisting}
	>>> s = "apple".ljust(10) + "orange".rjust(10) + "\n" \
	... + "grape".ljust(10) + "pear".rjust(10)
	>>> print s
	apple orange
	grape pear
\end{lstlisting}
\end{document}
